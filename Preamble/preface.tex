%
% Guidelines as of 2019/06/04
% https://gradresearch.unimelb.edu.au/__data/assets/pdf_file/0004/2027839/Preparation-of-GR-theses-rules.pdf
%
\Preface{

\addtocontents{toc}{\vspace{1em}}  % Add a gap in the Contents, for aesthetics

This preface includes a summary of all chapters in this work as well as a comprehensive summary of my contributions and everyone else's contribution. This is a thesis \textit{with} publications and each publication included in a chapter is shown here.

Solomon B.J\footnote{These authors contributed equally and are considered shared first.}., Tan L.\footnotemark[\value{footnote}], Lin J.J.\footnotemark[\value{footnote}], Wong S.Q.\footnotemark[\value{footnote}], \textbf{Hollizeck S.}\footnotemark[\value{footnote}], Ebata K., Tuch B.B., Yoda S., Gainor J.F., Lecia V. Sequist L.V., Oxnard G.R., Gautschi O., Drilon A., Subbiah V., Khoo C., Zhu E.Y., Nguyen M., Henry D., Condroski K.R., Kolakowski G.R., Gomez E., Ballard J., Metcalf A.T., Blake J.F., Dawson S-J., Blosser W., Stancato L.F., Brandhuber B.J., Andrews S., Robinson B.G., Rothenberg S.M
"RET Solvent Front Mutations Mediate Acquired Resistance to Selective RET Inhibition in RET-Driven Malignancies"
\textit{Journal of Thoracic Oncology.} 2020. DOI: \href{https://doi.org/10.1016/j.jtho.2020.01.006}{\nolinkurl{10.1016/j.jtho.2020.01.006}}

\textbf{Hollizeck S.}, Wong S.Q., Solomon B., Chandranada D.\footnote{These authors contributed equally and are considered shared last.}, Dawson S-J.\footnotemark[\value{footnote}] "Custom workflows to improve joint variant calling from multiple related tumour samples: FreeBayesSomatic and Strelka2Pass" \textit{Bioinformatics.} 2021. DOI: \href{https://doi.org/10.1093/bioinformatics/btab606}{\nolinkurl{10.1093/bioinformatics/btab606}}


\hyperref[ch:intro]{Chapter~\ref*{ch:intro}}: Introduction is an original work providing background and overview relevant to understanding the thesis and its relevance to the field. It includes an introduction to DNA, ctDNA, DNA sequencing, somatic variant calling and lung cancer.

\hyperref[ch:variantcalling]{Chapter~\ref*{ch:variantcalling}} Joint somatic variant calling is an original work describing two workflows for the joint analysis of multiple related tumour samples and has been published in \textit{Bioinformatics} as "Custom workflows to improve joint variant calling from multiple related tumour samples: FreeBayesSomatic and Strelka2Pass" on 21$^st$ September 2021. In addition to the published analysis, I have added longitudinal analysis and its evaluation.

Contributions for this chapter:
\begin{itemize}
	\item I conceptualised the work
	\item I implemented the workflows and containerised all required tools
	\item I performed the data simulation
	\item I performed the analysis presented in the publication
	\item I wrote the draft of the manuscript and performed revisions
	\item D.C. and S-J.D. provided advice in planning and writing the manuscript
	\item D.C. provided guidance for method development
	\item S-J.D. provided guidance for method evaluation
	\item S.W. performed the targeted amplicon validation
	\item S.W. and  B.S. read the draft version and provided feedback
	\item B.S. provided clinical expertise for human data
\end{itemize}

Where applicable, the following information must be included in a preface:
\begin{itemize}
\item[\tiny{$\blacksquare$}] a description of work towards the thesis that was carried out in collaboration with others, indicating the nature and proportion of the contribution of others and in general terms the portions of the work which the student claims as original;
\item[\tiny{$\blacksquare$}] a description of work towards the thesis that has been submitted for other qualifications;
\item[\tiny{$\blacksquare$}] a description of work towards the thesis that was carried out prior to enrolment in the degree;
\item[\tiny{$\blacksquare$}] whether any third party editorial assistance was provided in preparation of
the thesis and whether the persons providing this assistance are knowledgeable in the academic discipline of the thesis;
\item[\tiny{$\blacksquare$}] the contributions of all persons involved in any multi-authored publications or
articles in preparation included in the thesis;
\item[\tiny{$\blacksquare$}] the publication status of all chapters presented in article format using the
descriptors below;
    \begin{itemize}
        \item Unpublished material not submitted for publication
        \item Submitted for publication to [publication name] on [date]
        \item In revision following peer review by [publication name]
        \item Accepted for publication by [publication name] on [date]
        \item Published by [publication name] on [date]
    \end{itemize}
\item[\tiny{$\blacksquare$}] an acknowledgement of all sources of funding, including grant identification
numbers where applicable and Australian Government Research Training Program Scholarships, including fee offset scholarships.
\end{itemize}

FUNDING:

}
