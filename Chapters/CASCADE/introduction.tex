\section{Introduction}
\label{cascade-sec:intro}

\todo[inline]{talk about cascade autopsies}

\subsection{Lungcancer}
\label{intro-sec:lungcancer}

With around 1.6 million deaths world-wide each year, lung cancer is the number one cause of cancer death \cite{Siegel2018}. Every year about twelve thousand Australians get diagnosed with lung cancer. These cases can be generally split into two groups: small cell lung cancers (SCLC) and non-small cell lung cancers (NSCLC), which account for around 15\% and 85\% of cases, respectively. The majority of NSCLC are either lung adenocarcinoma or lung squamous cell carcinoma \cite{Molina2008}. Even though smoking is highly associated with lung cancers, there is a big group of never smokers, with a high risk of lung cancers in East Asia, especially women, which is correlated with outside influences like pollution and occupational carcinogens and paired with genetic susceptibility \cite{Sun2007}.
This group usually shows \textit{EGFR} (epidermal growth factor receptor) driven tumours. EGFR is a transmembrane receptor tyrosine kinase, which is usually only expressed in epithelial, mesenchymal, and neurogenic tissue, but its overexpression in other tissues is a hallmark of many human malignancies, not just NSCLC.


%% original abstract
%Approximately 50% of patients with non-small cell lung cancer (NSCLC) develop acquired resistance to EGFR tyrosine kinase inhibitors (TKIs) through mutations in EGFR T790M. Maintaining a dynamic balance between T790M positive and negative clones offers an opportunity to delay the emergence of resistance and improve disease outcomes. It is now possible to quantify EGFR mutations using circulating tumour DNA (ctDNA) which can provide a surrogate measure of clonal populations within tumours. This project will utilise next generation sequencing (NGS) of tumour tissue and ctDNA in patients treated with EGFR TKIs to study clonal evolution patterns and predict optimal treatment approaches to delay therapeutic resistance.