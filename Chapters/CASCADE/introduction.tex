\section{Introduction}
\label{cascade-sec:intro}

As tumour heterogeneity is seen as one of the major causes of resistance to treatment and ultimately relapse, much cell line based research has been conducted to solve tissue of origin and evolutionary trajectories via bulk and single cell sequencing paired with cellular barcoding \cite{Fennell2021,Penter2022}. However, while cell line models are a great resource for high throughput methods and allow easier reproducibility of results, they are no real substitute for primary patient cells. With the increased availability of patient samples through bio-banking efforts  like the UK BioBank \cite{Sudlow2015} and the Victorian Cancer BioBank \cite{CCV2006}, both patient derived xenografts (PDX) and organoids have gained more and more traction \cite{Yoshida2020} as specialised models to grow primary patient cells in an environment which closely resembles the body of the patient. While this method is superior in many aspects, there are some significant drawbacks. The culturing of the cells requires more effort and is not as easily scalable. These methods also require fresh patient samples, which are not always readily available.

While it is fairly easy to collect diagnostic specimens from tumour biopsies for storage and research, late stage tumour biopsies are rare. Due to the deteriorating health status of the patient biopsies can be dangerous and often an unnecessary burden for the patient. However, these samples are especially critical when answering the question of how the cancer was able to evade treatment and lead to death, as it may reveal an unappreciated insight into spatial and temporal heterogeneity.

To try to combat this issue the cancer tissue collection after death (CASCADE) program was initiated. It recruits cancer patients close to the end of life and enrolls them in a rapid autopsy program. These autopsies are carried out at any time of the day to minimise the impact on the sample to allow high quality assessment including DNA and RNA sequencing of the frozen cells \cite{Alsop2016}.
While the program collects cancer patients unconditionally of their type of disease, the analysis of this thesis is restricted to five lung cancer patients available at the time of this work. Currently there are no extra lung cancer patients enrolled, but recruitment is still ongoing. Four of these five patients had an Epidermal growth factor receptor (\textit{EGFR}) based cancer and one had a RET Proto-Oncogene (\textit{RET}) fusion with \textit{KIF5B}. Each of those patients had on average 30 specimens resected and put into a bio bank. We then continued to sequence, on average, eight of these samples with either whole genome sequencing (WGS) or whole exome sequencing (WES) to deeply analyse and classify the underlying resistance and driver mechanisms of each patient and their heterogeneity.

\subsection{Lung cancer}
\label{intro-sec:lungcancer}

With around 1.6 million deaths world-wide each year, lung cancer is the number one cause of cancer death \cite{Siegel2018}. Every year about twelve thousand Australians get diagnosed with lung cancer. These cases can be generally split into two groups: small cell lung cancers (SCLC) and non-small cell lung cancers (NSCLC), which account for around 15\% and 85\% of cases, respectively. The majority of NSCLC are either lung adenocarcinoma or lung squamous cell carcinoma \cite{Molina2008}. Even though smoking is highly associated with lung cancers, there is a big group of never smokers, with a high risk of lung cancers in East Asia, especially women, which is correlated with outside influences like pollution and occupational carcinogens and paired with genetic susceptibility \cite{Sun2007}.
This group usually shows \textit{EGFR} (epidermal growth factor receptor) driven tumours. EGFR is a transmembrane receptor tyrosine kinase, which is usually only expressed in epithelial, mesenchymal, and neurogenic tissue, but its overexpression in other tissues is a hallmark of many human malignancies, not just NSCLC.

Even with those strict classifications in place, it is widely accepted, that cancer is a heterogeneous disease, which needs to be accounted for when developing treatments \cite{Suda2018}. The ongoing research of lung cancer has led to a shift from cytotoxic chemotherapy to a more personalized approach by accounting for the genetic background of each patient’s disease \cite{Lindeman2018}. 
But not only the inter-patient heterogeneity needs to be taken into account, but also the heterogeneity between different sites of the disease in the same patient \cite{Leong2018,Savas2016}. This makes the choice of treatment for one single patient more and more difficult, as some sites might respond to treatment, where others might not. This means, in order to design the perfect treatment regime for a patient, a deep understanding of the overall complexity of the disease is needed. By studying a diverse background of driver mechanisms of lung cancers and their respective treatment and resistance modes, a general insight in the biologic background is possible. Analysing not only one, but several metastases of the same patients paints a much clearer picture of disease progression and the process behind the resistance to treatment that ultimately led to death. 


%% original abstract
%Approximately 50% of patients with non-small cell lung cancer (NSCLC) develop acquired resistance to EGFR tyrosine kinase inhibitors (TKIs) through mutations in EGFR T790M. Maintaining a dynamic balance between T790M positive and negative clones offers an opportunity to delay the emergence of resistance and improve disease outcomes. It is now possible to quantify EGFR mutations using circulating tumour DNA (ctDNA) which can provide a surrogate measure of clonal populations within tumours. This project will utilise next generation sequencing (NGS) of tumour tissue and ctDNA in patients treated with EGFR TKIs to study clonal evolution patterns and predict optimal treatment approaches to delay therapeutic resistance.