\section{Discussion}
\label{cascade-sec:outlook}

In this chapter we described both the high inter and intra patient heterogeneity of late stage lung cancer patients and showed that mitochondrial variant based phylogeny could help to resolve the sample relationships in the context of selection pressure through treatment. Additionally we uncovered a different disease trajectory for cases showing SCT where the cancers appeared genetically primed for SCT, but genetic alterations alone were on sufficient to drive transformation. This suggests that genetic analysis alone will not allow the prediction or early detection of SCT as a resistance mechanism. This uncoupling of genetic evolution and disease histology was also reported recently in a study focusing on multi-region analysis of treatment na\"ive SCLC cases~\cite{Zhou2021}.

While there exist several multi-region lung cancer studies up to date, their focus was on early stage and treatment na\"ive disease \cite{JamalHanjani2017,Leong2018}. While these studies showed ubiquitous intra-tumour heterogeneity and copy number alterations in early stage disease, there is an unmet need for the assessment of late stage lung cancers\add{, especially with the ability of a full body work-up and not just limited to surgically resected specimens }\cite{Zhang2014}. \add{ As surgical specimen provide very high quality samples, almost all large scale studies rely on these early-stage lung cancers}\cite{Network2012,CGARN2014,George2015}\add{, which is still considered ground truth for studying tumour heterogeneity in lung cancer. However, this lack of insight into in advanced and late-stage cancer heterogeneity may affect the development of effective target therapies for these cancers.}

With this work we took a first step towards understanding and measuring the heterogeneity of tumour samples and treatment resistance mechanisms in late stage NSCLC, but many unanswered questions remain. The epigenetic marks and their inheritance patterns in cancer are a massive unexplored field which increases the heterogeneity of cancer even more \cite{Easwaran2014}. Additionally, we could only hypothesise and reconstruct the longitudinal trajectory of the disease from autopsy samples. The next step to validate and further explore these findings would be to analyse temporally spaced samples from the same disease.