\section{Outlook}
\label{cascade-sec:outlook}

In this chapter we described both the high inter and intra patient heterogeneity of late stage lung cancer patients and showed that mitochondrial phylogeny could help to resolve the sample relationships in the context of selection pressure through treatment. Additionally we uncovered a different disease trajectory of SCLC where the cancer is genetically already transformed, but has not changed histological. This suggested that the early detection of SCT requires the study of pre-transformed SCLC like patient CA-L, to not detect symptoms of the disease and rather the first signs of a potential transformation.

With this work we took a first step towards understanding and measuring the heterogeneity of tumour samples, but many unanswered questions remain. The epigenetic marks and their inheritance patterns in cancer are a massive unexplored field which increases the heterogeneity of cancer even more \cite{Easwaran2014}. Additionally, we could only hypothesis and reconstruct the longitudinal trajectory of the disease from autopsy samples. The next step to validate these findings and explore the hypothesis is to analyse temporally spaced samples from the same disease.