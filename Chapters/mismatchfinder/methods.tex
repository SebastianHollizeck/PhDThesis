\section{Methods}
\label{mmf-sec:methods}

With the change from a variant focused approach to a read based method, this new method will call ``mismatches`` of a read from the reference genome, rather than a variant. This has the advantage of not requiring a matched normal and its use for virtually any sequencing data source, be it TAS, WES, WGS or even nanopore sequencing\footnote{however nanopore is not really usefull due to the short fragments naturally occuring in cfDNA}. However it also means, that the error supression method, which are usually used by variant calling methods like read position ranks sum (RPRS) or strand bias are not useable, which leads to a higher degree of background noise. In the following sections I will describe how we filter and curate the found mismatches to retain as much signal as possible.

\subsection{Data preprocessing}
As this new method has sophisticated internal measures to filter and process sequencing data, the steps for preprocessing are minimal: The reads only need to be aligned to a reference genome (\autoref{intro-sec:mapping}). For optimal mapping and additional noise reduction, paired end sequencing of at least 75 bp is suggested. This ensures a few bases overlap on the standard fragment length of less than 150b of ctDNA (\autoref{intro-sec:ctDNA}). Another optional suggested step is the duplication marking of the BAM file.

\subsection{Mismatch detection}
In contrast to conventional variant calling approaches, which find regions of interest through pileups (position wise) and then realign reads in the surrounding area, to accurately estimate the most likely event that lead to the observed haplotype (\autoref{intro-sec:variantcalling}), with this new method, we take every individual read as a separate entity to fully span the heterogeneity of all cells and their genetic background. A sequencing reads ``MD``- and ``CIGAR``- tag from the preprocessed BAM file are used to reconstruct the sequence of the read and the positions, where the read shows a different base than the reference. These potential mismatch sites will then filtered in multiple steps to reduce the impact of both germline variants as well as sequencing errors

\subsection{Filtering steps}
Apart from the filters, which most variant callers will employ, like mapping quality (MQ) and base quality (BQ), which are used to ignore reads as well as positions respectively, the method also internally filters out common sequencing errors next to homopolymer regions \cite{Heydari2019}. While these cutoffs were preselected by me for optimal performance on our data (MQ=20, BQ=55, homopolyLength=5), the program allows the user to adjust them to their liking.
This is also possible for both the region of interest (ROI) bed-file which was used to restrict the analysis to only highly mappable regions of the genome (\autoref{ch-mmfAppendix:bedfiles}), as well as for multiple other parameters which are unique to our method, like minimum average base quality, minimum and maximum number of mismatches per read and/or fragment, and the minimum and maximum length of a fragment. If any of these values are not within the specified range a read will be discarded in the analysis. This is also the default for reads which have a secondary alignment position or are considered duplicates of any kind.

\subsection[Consensus reads]{Consensus reads - what happens when the sequencer isn't sure}
\label{mmf-sec:consensus}

\todo[inline]{write about the consensus building process with some figures}