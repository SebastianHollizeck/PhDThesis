\section[Longitudinal analysis]{Longitudinal analysis - something for the ages }
\label{variantcalling-sec:longitudinal}

\todo[inline]{Consider the order of longitudinal vs downstream}

While the initial motivation for the development of these tools was the analysis of multi-region, so spatial, samples from the same patient coming from the CASCADE rapid autopsy program, a longitudinal application of these methods for the joint analysis of diagnostic and relapse sample, or even the repeated testing of ctDNA are quite worth thinking about. In this part, I will present work using the published workflows on other datasets, which highlights the flexibility and wide spread use of our new methods.

Specifically, I will show the analysis of longitudinal ctDNA WES of patient ``CA-F`` from the manuscript. in a study of late stage melanoma patients, \Citeauthor{Tan2019} identified ctDNA sequencing as a way to stratify patients into high and low risk of relapse and therefore inform adjuvant therapy \cite{Tan2019}. In this analysis we aim to improve the quality of detection of low allele frequency somatic variants. This would enable the detection of variants from brains metastasis, as the blood brain barrier decreases the availability of DNA fragments from lesions in the brains in the normal blood stream \cite{2014}, however it is detectable with sensitive enough methods \cite{Yoon2019,Ma2020}.

\todo[inline]{select the right dataset to show here (possibly the one from the MisMatchFinder stuff)}