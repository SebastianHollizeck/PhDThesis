\section[Effects on downstream analysis]{Effects on downstream analysis - not quite the missing link, but close}
\label{variantcalling-sec:downstream}

The ability to find additional shared variants has significant impact on our understanding of cancer evolution and the timing of initiation and metastatic seeding. Recent work has shown, that similar to the well known genetic heterogeneity, there is heterogeneity when it comes to metastatic seeding. While traditionally it was thought that tumours only metastesised after they reached a certain size, to escape the restrictions of the niche, like reduced nutrition, recent publications showed, there is also very early metastatic seeding \cite{Hu2019}. 
But all those methods are ultimately based on the somatic variants found in the data, so if we improve on the input of the downstream analysis methods, we can expect a clearer and possibly more granular result.

In this section I will highlight for a few examples on how big the effect can be for methods like phylogenetic reconstruction and clonal decomposition.

\subsection[Polygenetic reconstruction]{Phylogenetic reconstruction}
\label{variantcalling-sec:phylo}

\todo[color=orange]{show before vs after for phylogenetic reconstuction}


\subsection[Clonal deconvolution]{Clonal deconvolution}
\label{variantcalling-sec:clonal}

\todo[color=orange]{show before vs after for clonal deconvolution}