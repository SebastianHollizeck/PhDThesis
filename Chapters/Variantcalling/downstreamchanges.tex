\section[Effects on downstream analysis]{Effects on downstream analysis - not quite the missing link, but close}
\label{variantcalling-sec:downstream}

The ability to find additional shared variants has significant impact on our understanding of cancer evolution and the timing of initiation and metastatic seeding. Recent work has shown, that similar to the well known genetic heterogeneity, there is heterogeneity when it comes to metastatic seeding. While traditionally it was thought that tumours only metastesised after they reached a certain size, to escape the restrictions of the niche, like reduced nutrition, recent publications showed, there is also very early metastatic seeding \cite{Hu2019}. 
But all those methods are ultimately based on the somatic variants found in the data, so if we improve on the input of the downstream analysis methods, we can expect a clearer and possibly more granular result.

In this section I will highlight for a few examples on how big the effect can be for methods, like phylogenetic reconstruction and clonal decomposition, which use somatic variants as input.

\subsection[Polygenetic reconstruction]{Phylogenetic reconstruction}
\label{variantcalling-sec:phylo}
As this work is not about the advantages and shortcomings of different phylogenetics reconstruction tools, we will not show a comprehensive amount of these tools, but rather focus on the results.  For this reason, we chose to use neighbour joining (NJ) \cite{Saitou1987}, because it is fast, readily available in most phylogenetic reconstruction toolkits and if the input distance is correct, the output will be correct. And even, if the distance is not 100\% correct, if the distance is ``nearly additive`` and the input distances are not far off the real distance, the tree topology will still be reconstructed correctly \cite{Mihaescu2007}. Lastly, in contrast to many other methods like UPGMA and WPGMA \cite{Sokal1958}, NJ does not assume an equal mutation rate of each sample, because we know, that the molecular clock hypothesis \cite{Zuckerkandl1962} is not valid for different lineages of cancers \cite{Shibata2010}.

While there are lots of distance measures for DNA sequences, which allow accounting for different probabilities of transitions and transversions as well as uneven base composition, models like F81 \cite{Felsenstein1981} or HKY85 \cite{Hasegawa1985} are only really designed for germline mutations and are not easily applicable for subclonal somatic mutations, which is why we decided to first transform the variants present in all samples into a binary occurrence vector and then calculating the Hamming distance \cite{Hamming1950} between all samples. This generates a maximum parsimony approach and the branch length of the trees will be directly translatable to the amount of variants which are different between samples. 
\todo[color=orange, inline]{show before vs after for phylogenetic reconstuction}


\subsection[Clonal deconvolution]{Clonal deconvolution}
\label{variantcalling-sec:clonal}

\todo[color=orange, inline]{show before vs after for clonal deconvolution}