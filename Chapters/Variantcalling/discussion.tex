\section[Discussion]{Discussion}
\label{variantcalling-sec:discussion}

\note{Added discussion}
In this chapter, we presented published work of workflows for the joint analysis of multiple related tumour samples. These two workflows are specifically designed and tuned to improve the analysis of multi-sample cancer datasets, which at the moment are all analysed in a tumour-normal pairwise method. We show that combining the analysis instead of the results drastically increases synthetic and clinical data sensitivity. However, this sensitivity increase is limited to variants shared by at least two samples. For private variants, variants unique to a sample, the workflows will perform approximately the same as the pairwise methods they are based on. The workflows are slightly more susceptible to noise generated through technical artefacts, as reoccurring biased errors could be interpreted as a genuine signal in the joint analysis. This error can be seen in the drop in precision in the synthetic dataset. However, in our use case and datasets, the positives outweighed the negatives.

The benefits were highlighted in the additional work we have conducted using clinical spatial and longitudinal data, where our methods enabled a more granular downstream analysis with phylogenetic reconstruction and clonal deconvolution. While the results from the pairwise analysis could be accurate, the joint analysis results are more compatible with the clinical progression and prior biological knowledge of late-stage cancers. Ultimately, joint somatic variant calling method development requires fully resolved truth sets, akin to the genome-in-a-bottle efforts to be validated completely. However, these truth sets do not exist for tumour-normal pairs yet, so variant call validation with TAS, as we performed it, is required to evaluate new algorithms and workflows.

While these methods are performing well on our data, they should be considered a proof of concept, and the concepts picked up by future developments in the variant calling space, especially in the single-cell field, where the sparse data of thousands of cells could benefit from a joint variant calling approach as shown in this chapter.