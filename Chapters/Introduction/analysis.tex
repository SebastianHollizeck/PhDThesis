\section[DNA analysis]{DNA analysis - what to do with the sequence}
\label{intro-sec:analysis}
The types of analysis that can be done with the output from the sequencing machine stretches far,  however, all methods need to first infer the location in the genome, the sequenced piece of DNA originated from. As the current methods randomly fragment the DNA (\autoref{intro-sec:libraryprep}), the genomic location information is completely lost. This process is referred to as mapping.

\subsection[Mapping]{Mapping - inferring genomic location of a read}
\label{intro-sec:mapping}
In this process, the fragments of DNA, which were sequenced, are assigned a genomic coordinate on the reference genome. This is only possible, due to the fact, that we have a resolved genome sequences (\autoref{intro-sec:sequencing}) for a high number of species. The location a sequenced piece of DNA matches to the reference genome might be unique, but it could also align to multiple locations, due to highly repetitive regions or due to the existence of pseudo genes with almost 100\% identity concordance. In addition to this, the reference genome might not accurately reflect the genome of the organism that has been sequenced. Each mapping position is therefore assigned a quality score, which reflects how likely it is to be the actual position of the sequence. As Illumina sequencers have the ability to sequence both ends of the DNA fragment, the position of the ends (read 1 and read 2) relative to each other can also be used to infer the quality, as they should be within a reasonable distance to each other (\autoref{fig:libraryprep})

As this process is time-consuming and the exact location of the fragment might not be as important, there exists a subset of tools called pseudo-mappers, which are based on $k$-mers, which are predefined DNA sequences of length $k$, which help to identify certain regions of interest. These tools are especially common for RNAseq, where the exact location of a read does not matter, only that the read is within a gene \cite{Bray2016,Patro2017}, but also for methods that estimate similarity between sequences (DNA, RNA or protein) \cite{Ondov2016,Luczak2017}.

For this work however, the exact position of reads is crucial, so only precise mapping methods like BWA \cite{Li2013} or Bowtie 2 \cite{Langmead2018}, which are optimised for short reads from Illumina systems, provide the necessary functions.


\subsection[Variant calling]{Variant calling - ``Spotting the difference``}
\label{intro-sec:variantcalling}
As intra-species genetic variation is intended for adaptation and evolution, there will be places where the DNA sequence of the subject will differ from the sequence of the reference (see \autoref{intro-sec:mutations}). In the context of cancer, these variants can give insights into the development and progression of cancer, and treatment options for patients and can even be used to guide family planning. Depending on the type of variation that is of interest, a different set of computational methods are needed, as germline and somatic variants have different properties.

\subsection[Germline]{Germline variant calling - the cards you have been dealt at birth}
\label{intro-sec:germlinecalling}
The most common source of DNA used for germline variant analysis is the mono nuclear layer from the blood of the subject, but really almost any cell can be used for this process, as all cells in the organism will share all germline variants (\autoref{intro-sec:mutations}). The only important input on top of the DNA sequence from the sequencer are the reference genome of the organism, as all variant nomenclature is based on the reference and the ploidy of the organism (\autoref{intro-sec:ploidy}). The ploidy is key to understand the ranges of allele frequencies at which a variant can biologically occur. For example in a human diploid genome, germline variants can occur either in one or both chromosomes, which mean we assume reads should show an allele frequency of around 50\% and 100\%, whereas the hexaploid commercial wheat \cite{Mayer2014} allele frequency for variants would be 16\%, 33\%, 50\%, 66\%, 0.83\% and 100\%. Due to the random sampling and possible sequencing errors, the observed allele frequencies may differ from the theoretical values. 
Most state of the art germline variant calling methods will also use haplotype reconstructions through de-Bruijn graphs, which features a remapping of reads in relation to each other \cite{Garrison2012,Lai2016,Kim2018,Benjamin2019,Cooke2021} where the original mapping location assigned by the aligner (\autoref{intro-sec:mapping}) is only used as a guideline. This allows one to resolve even complex haplotypes of the sample by not restricting the method to the linear setup of the reference genome.


\subsection[Somatic]{Somatic variant calling - life is ever-changing}
\label{intro-sec:somaticcalling}
In contrast to germline variant calling, somatic variant calling methods cannot rely on allele frequency, as not all cells sequenced are expected to have the change in nucleotide. The allele frequency is instead a measure of the sub clonal size. A subclone is here defined as the set of cells, which were derived from the cell, which originally acquired the somatic mutation. Depending on the selective advantage, random drift and also the time point when the variant was introduced, these clones can be very variable in size and therefore their contribution within the DNA sequenced.
As not all cells have the variants, the selection of the tissue for library preparation is very important, unlike for germline calling. 
In the setting of cancer, somatic variant calling is important in understanding the genetic landscape of caner samples, where the main question is, which changes are present in the tumour and which lead to disease.

The ideal scenario for tumour somatic variant calling is when a biopsy of the tumour as well as a normal sample of the patient is available. In most clinical cases, this will be the diagnostic biopsy as well as the mono nuclear layer from blood, so true somatic variants can be chracterised (\autoref{intro-sec:germlinecalling}). 
These two samples are then analysed together and only changes that are only in the somatic tumour sample and not in the normal sample are reported. Even though this concept sounds simple, there are some pitfalls \cite{GATKTeam2021a}. First, there might be some tumour ``contamination`` in the normal sample, which needs to be adjusted for \cite{Kim2018,TaylorWeiner2018}. Second, there might be normal ``contamination`` in the tumour sample, which means that not all cells in the tumour sample are actually tumour. This means that the signal of the tumour specific changes may be reduced and harder to find.

All of these issues are amplified in the scenario, when there is no ``normal`` sample available, either because the patient did not consent, due to other medical issues, or because for diagnostic tests thereis often no need for a germline sample. In this case, there is the option for ``tumour only`` variant calling, which utilises a database of germline variants in the population, to distinguish between somatic and germline variants, as the variant calling is very similar to just germline variant calling (\autoref{intro-sec:germlinecalling}) without the restriction of the ploidy. However, even with an extensive database like gnomAD \cite{Karczewski2020} it is unlikely to be able to remove all germline variants from the analysis and as there is no direct comparison, the precision of the ``tumour only`` method is significantly lower \cite{Karimnezhad2020}.
