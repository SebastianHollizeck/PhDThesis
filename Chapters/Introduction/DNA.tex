\section[DNA]{DNA as a information storage unit}
\label{intro-sec:DNA}
It is a widely accepted fact, that Deoxyribonucleic acid (DNA) serves as the long term information storage molecule of our cells. This information is protected and allows correction of simple errors through its double helix structure \cite{Watson1953,Liang1998}. The nucleotides, which consist of a deoxyribose sugar (hence the name), a phosphate group and the nitrogenous base, are joined together by phosphate groups. Even though there are six common naturally occurring nitrogenous bases: adenine (A), thymine (T), guanine (G), cytosine (C), uracil (U) and nicotinamide, only the first four are used to encode the genetic information into DNA. Each of the strands mirrors the other, so that an adenine will be paired up with a thymine forming two hydrogen bonds. Similarly cytosine will pair with guanine forming an even stronger bond with three hydrogen bonds. While other pairings which do not follow those rules are chemically possible, they are mostly observed in ribonucleic acid (RNA) \cite{Sinden1994}. These very strict bonding rules enable the DNA to be similar to a hard drive with backup on a computer. And as only one strand contains all the information, the DNA polymerase enzyme does only need access to one strand, which allows parallel replication during cell division, but also error corrections, by proof reading the newly synthesised strand with the template.
The DNA in eukaryotes however is not free floating around in the nucleus of a cell, but rather it is highly organised around histones, which then form something resembling a spool of thread. This allows some of the DNA to be accessible where the use of other areas can be restricted. Through this restriction, the availability of certain genes, which are the sections of the DNA, which encode for short term storage molecules like RNA. This restriction plays an important role in cell fate and cell viability. Ultimately all information stored to create a new highly complex organism is stored in just the DNA of one cell. Whichever parts are used and how they are used decides the function and the identity of the cell.
\subsection[Mutations]{Phantastical mutations and where to find them}
However even though the DNA is highly stable and error correction methods are constantly working to not introduce any changes in the DNA, the source of evolution and adaptation of species is sourced in a steady mutation rate. These changes in normal tissue are mostly irrelevant to the organism as a whole and will not be passed on to the next generation. These changes are known as somatic mutations. This type of mutation accumulates in a cell linearly over the course of the lifespan of the cell and is not bound to just cell divisions\cite{Alexandrov2015,Moore2021}. 
In contrast, if one of those mutations occurs in the germline cell, eg. sperm or egg producing cells, these mutations will be propagated to all offspring and be present in all cells of that organism and in term all its offspring. These mutations are called germline mutations. These mutations are also called germline variants, as they establish in the population and represent a variation of the organism.