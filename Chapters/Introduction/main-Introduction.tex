% main introduction

\begin{savequote}[85mm]
``Begin at the beginning,`` the King said, very gravely, ``and go on till you come to the end: then stop.``
\qauthor{--- Lewis Carroll, \textit{Alice in Wonderland}}
\end{savequote}

\chapter{Introduction}
\label{ch:intro}

%\epigraph{``Begin at the beginning,`` the King said, very gravely, ``and go on till you come to the end: then stop.``}{ --- \textup{Lewis Carroll}, Alice in Wonderland}


This first introduction chapter contains all the necessary background information as well as an overview for the work discussed in this thesis. It summarised basic biological properties of DNA and cell biology as well as the respective technologies to read, analyse and measure these biological concepts and then how to evaluate the output of these methods.
\hyperref[intro-sec:DNA]{Section~\ref*{intro-sec:DNA}} delineates the role DNA plays for the cell and then \hyperref[intro-sec:ctDNA]{section~\ref{intro-sec:ctDNA}} shows how these standards are changed in the tumour and cell free context. \hyperref[intro-sec:sequencing]{Section~\ref{intro-sec:sequencing}} introduces the current technologies used to measure and detect DNA and its variations. With \hyperref[intro-sec:analysis]{section~\ref*{intro-sec:analysis}} covering the computational analysis methods to read out changes in the DNA. Then \hyperref[intro-sec:lungcancer]{section~\ref{intro-sec:lungcancer}} relates how these changes lead to cancer and what we can learn from them. 
The introduction concludes with \hyperref[intro-sec:overview]{section~\ref*{intro-sec:overview}} as an overview over the thesis aims and my work in addressing them in the following chapters.


%%% this just contains all the links and possible formating of the introduction as a whole

% background of DNA
\section[DNA]{DNA as a information storage unit}
\label{intro-sec:DNA}

\begin{figure}[!ht]
\centering
\includegraphics[width=0.9\linewidth]{Figures/DNAStructure}
\caption[Overview DNA structure]{Overview of DNA structure and the nucleobases, which form DNA strands. Nucleotides are split into Purines and Pyrimidines by the structure of the nitrogen ring; complementary pairing of bases is shown as shapes of the bases as well as with 2D structures; Hyrdogen (H) bonds are shown as dotted lines; Phosphates are shown as P; 3' and 5' ends are defined by the internal number of the carbon atom of the sugar which is exposed; Adapted from ``DNA structure`` by \href{https://biorender.com}{\nolinkurl{BioRender.com}} (2021) Retrieved from \href{https://app.biorender.com/biorender-templates}{\nolinkurl{https://app.biorender.com/biorender-templates}}}\label{fig:DNAstructure}
\end{figure}


It is a widely accepted fact, that Deoxyribonucleic acid (DNA) serves as the long term information storage molecule of our cells. This information is protected and allows correction of simple errors through its double helix structure \cite{Watson1953,Liang1998}. The nucleotides, which consist of a deoxyribose sugar (hence the name), a phosphate group and the nitrogenous base, are joined together by phosphate groups. Even though there are six common naturally occurring nitrogenous bases: Adenine (A), Thymine (T), Guanine (G), Cytosine (C), Uracil (U) and nicotinamide, only the first four are used to encode the genetic information into DNA. Each of the strands mirrors the other, so that an adenine will be paired up with a thymine forming two hydrogen bonds. Similarly cytosine will pair with guanine forming an even stronger bond with three hydrogen bonds. While other pairings which do not follow those rules are chemically possible, they are mostly observed in ribonucleic acid (RNA) \cite{Sinden1994}. These very strict bonding rules enable the DNA to be similar to a hard drive with backup on a computer. And as only one strand contains all the information, the DNA polymerase enzyme does only need access to one strand, which allows parallel replication during cell division, but also error corrections, by proof reading the newly synthesised strand with the template. In order to be able to distinguish the two strands, they were assigned the names 3' and 5' depending on the numbering of the carbon atom in the sugar, which is exposed (\autoref{fig:DNAstructure}).

The entirety of the DNA encoding the organism is commonly called ``the genome`` with all genes, which consist of introns and exons are called exome. Unicellular organisms usually only have a very small amount of introns, which to current knowledge only provide limited information and are only responsible for structure. In vertebrates  introns as well as intergeneic DNA (the DNA between genes) contribute most of the DNA in the genome. For example in humans, only $1\%$ of the genetic material is considered to be exonic, whereas introns contribute $\approx 24\%$ and the rest is intergeneic ($\approx 75\%$)\cite{Venter2001}.

\begin{figure}[!ht]
\centering
\includegraphics[width=0.9\linewidth]{Figures/ChromosomeStructure}
\caption[Overview Chromosome structure]{Structural overview of the metaphase condensed chromosome: DNA is first wrapped around Histones to form nucleosome, which then associate with each other to form the chromatin fiber, which in the metaphase of the cell cycle is condensed even more into the X-shaped chromosome}\label{fig:chromsomestructure}
\end{figure}

The DNA in eukaryotes however is not free floating around in the nucleus of a cell, but rather in most eukrayotic organisms, it is highly condensed and structured, first wrapped around nucleosomes like thread on a spool, then organised around histones, into either open (accessible) or closed chromatin, which then can be even further condensed into chromosomes, which have a X-like shape, with two shorter and two longer arms (\autoref{fig:chromsomestructure}). This allows some of the DNA to be accessible where the use of other areas can be restricted\cite{Hammond2017}. Through this restriction, the availability of certain genes, which are the sections of the DNA, which encode for short term storage molecules like RNA. This restriction plays an important role in cell fate and cell viability. Ultimately all information stored to create a new highly complex organism is stored in just the DNA of one cell. Whichever parts are used and how they are used decides the function and the identity of the cell\cite{Bonev2016}. 



\subsection[Ploidy]{Ploidy - its good to have a backup, if you do it right}
\label{intro-sec:ploidy}
Similar to the already discussed RAID-like setup of the DNA in two strands, another concept of data security, a spatial different storage is also implemented. Most eukaryotic organisms have at least two of each chromosome (diploid) with some species reaching up to septaploid\cite{Tateoka1975}. However, this concept is not the only reason for the ploidy of somatic cells. For sexually reproducing organisms, at least a diploid set of chromosomes is necessary to enable information to be joined from both parents. Germline cells (sperm and egg) are generally monoploid, such that the resulting cell will be diploid, but the ploidy of the somatic cells is not as uniform within a species, where it can vary between organisms based on gender or rank \cite{Trivers1976}. 
In most organisms, a change in ploidy is fatal \cite{Otto2007} and only partial ploidy changes like extra copies of chromosome 17 \cite{Gottlieb1962}, chromosome 18 \cite{Cereda2012} and chromosome 21 \cite{Hulten2008} are tolerated. These syndroms can occur when the is an uneven split of chromosomes during cell division.
The additional advantage, apart from sexual reproduction, is that a second almost identical copy of a chromosome allows repair of DNA, even when both strands are damaged, for example in a double strand break.
In this case, the information from the sister chromosome will be used, by first cutting the double strand break ends to have overhang (resection). This overhang will then merge with the sister chromosome's mirrored strand. In this state, the two chromosomes are fused together in a Holliday junction, which allows the missing part from the resection and the double strand break to be synthesised \cite{Lilley2000}. During this process, which is part of the homology directed repair (HDR) machinery, the sister chromosomes exchange parts of their DNA, when resolving the Holliday junction. As these stretches of DNA do not need to be 100\% identical, this plays and important role in evolution and diversity \cite{Hanage2006,Kong2013}.

\begin{figure}[!ht]
\centering
\includegraphics[width=0.9\linewidth]{Figures/ChromosomeTerritories}
\caption[Overview DNA structure]{Individual chromosomes occupy a subspace in the nucleus called chromosome territories. Chromosome territories can be further partitioned to distinct A and B compartments, which are enriched for active and repressed chromatin, respectively. Genomic regions within topologically associating domains (TADs) display increased interactions, while their interactions with neighbouring regions outside of the TADs are rather limited.}\label{fig:chromosometerretories}
\end{figure}

Even though this X-like structure is the most commonly used and known structure, the DNAs 3D structure is usually very different and only takes this shape for the very short time of the cell cycle. Most of the time, the chromosomes are unravelled into something resembling a ball of yarn, where the ``open`` chromatin regions are on the outside and the ``closed`` regions are ``hidden`` in the inside and each chromosome establishes its own ``territory`` inside the nucleus (\autoref{fig:chromosometerretories}). This structure allows another DNA cross over with non-sister chromsomes, which is called a chiasma.

\subsection[Mutations]{Phantastical mutations and where to find them}
\label{intro-sec:mutations}
However even though the DNA is highly stable and error correction methods are constantly working to not introduce any changes in the DNA, the source of evolution and adaptation of species is sourced in a steady mutation rate \cite{Darwin2010,Sprouffske2018}. These changes in normal tissue are mostly irrelevant to the organism as a whole and will not be passed on to the next generation. These changes are known as somatic mutations. This type of mutation accumulates in a cell linearly over the course of the lifespan of the cell and is not bound to just cell divisions\cite{Alexandrov2015,Moore2021}. 
In contrast, if one of those mutations occurs in the germline cell, eg. sperm or egg producing cells, these mutations will be propagated to all offspring and be present in all cells of that organism and in term all its offspring. These mutations are called germline mutations. These mutations are also called germline variants, as they establish in the population and represent a variation of the organism.
Mutations can also be classified depending on either their size ranging from single nucleotide polymorphisms (SNPs) over small insertions or deletions (InDels) to large structural changes, like the deletion of parts of or even a whole chromosome arm. like previously described with ploidy changes, usually smaller changes have less impact on the overall fitness of the organism, however even SNPs can lead to changes which are not compatible with life\cite{Shamseldin2015,Frey2021}.



% background of ctDNA
\section[cfDNA]{Cell free DNA is more than just bits and pieces}
\label{intro-sec:ctDNA}

When a cell from a multicellular organism dies, through which ever method, there will be numerous enzymes involved, which clear the debris and recycle material. This means that proteases digest proteins into amino acids, which will later be used for either building new proteins or possibly even digested further for energy production. The same happens with the DNA in the cell. However, as discussed in the previous section~\ref{intro-sec:DNA} the DNA is wrapped around histones and organised in structures called nuclesomes. These protect the DNA from being cut by DNAases by hindering the access to the DNA, similar to how they stopped the access for transcription into RNA. This then in turn leads to the DNA being cut into pieces mainly in the length of 167 base pairs (bp). 
These DNA fragments, which are called cell free DNA (cfDNA), can then be detected in bodily fluids, like blood or even stool. By analysing these fragments, non invasive tests for prenatal care have been possible, as the DNA of the foetus is detectable in the mother's blood \cite{Dan2012,Nicolaides2013}.
Similar to the process, a cancer also sheds DNA, titled circulating tumour DNA (ctDNA), when its cells die, either through intervention of the immune system or through other forcefull processes. These ctDNA fragments can also be analysed and molecular properties measured, without even knowing the exact location of the tumour. As a blood test can be routinely performed in the clinic or even a general practitioner, the monitoring of cancer progression is significantly easier and safer than through other measures. Of course, it is, similar to the prenatal test, only a proxy for the cells which are still alive, as these have not shed their DNA. Additionally, the amount of shedded DNA is highly variable between tumours, with a general higher amount for later stages, so that sometimes there is almost no ctDNA present, even though the cancer is fairly advanced \cite{Diehl2008,Schwarzenbach2011}.

\todo[inline]{include the length of ctDNA}

% background of sequencing DNA
\section[DNA sequencing]{DNA sequencing - when is next generation sequencing the current generation?}
\label{intro-sec:sequencing}

As we know the building blocks that make DNA \change{as well as}{and} the processes and the enzymes responsible, we can synthesise DNA in vitro. By chemically modifying the nucleotides supplied to the synthesis process, the sequence of the copied strand can be analysed. The first method \remove{to make use of this }used the lambda phage to fuse known ends for the primers needed for the reaction to the piece of DNA and supplied labelled nucleotides~\cite{Padmanabhan1974}. This method was then superseded by "Sanger sequencing" after Frederick Sanger, who, with colleagues, published this method in 1977. Through adding \textbf{di}deoxynucleotides in a low concentration, the polymerase chain reaction would terminate trying to integrate these nucleotides, and by labelling them radioactively or fluorescently, a gel could then be used to determine the sequence of the piece of DNA \cite{Sanger1975,Sanger1977}, which made the method better suited for large-scale projects.

However, this method had multiple issues for modern research questions. Mostly, \remove{that }it was fairly labour-intensive and time-consuming to analyse multiple pieces of DNA \change{at the same time}{simultaneously}, making it very challenging to sequence all the DNA of an organism. The human genome project, which was started in 1990, used machines that automated the Sanger sequencing procedure, and it still took hundreds of researchers 13 years to complete the DNA sequence of just one human \cite{Lander2001,Venter2001}. Even though this was a very long project, it laid the groundwork for \change{the usage of}{using} the current sequencing technologies.

\subsection[Library preparation]{Library preparation - what we learned from using phages}
\label{intro-sec:libraryprep}

\begin{figure}[htb]
\centering
\includegraphics[width=.9\linewidth]{Figures/intro/LibraryPreparation.png}
\caption[Library preparation for NGS]{Adapter ligation during library preparation. The adapters are added to the DNA insert during library preparation. A. The DNA insert is prepared by adding an A-tail and phosphorylation. B. The adapter complex which includes the P5/P7 flow cell binding adapter is added to the DNA insert. C. The DNA insert is ready for sequencing. D. The DNA insert binds to the flow cell for sequencing. Primers bind to the DNA insert to generate reads; Figure adapted from \href{https://sapac.support.illumina.com/bulletins/2020/12/how-short-inserts-affect-sequencing-performance.html}{"How short inserts affect sequencing performance"}~\protect\cite{Illumina2020}}\label{fig:libraryprep}
\end{figure}

Library preparation is the name of the preprocessing step, which is done before it is sequenced with the current technologies. The first step to sequence DNA is to obtain the DNA, which is done by lysing the cells of interest, which disrupts the cell membrane and therefore spills all its contents. The then spilled DNA is fragmented into smaller pieces, by either restriction enzymes or sonication, to have a size of about between 200-800bp. These steps are not necessary when preparing \add{the} sequencing of ctDNA, as discussed in \autoref{intro-sec:ctDNA}, as the DNA is unbound and already digested into short fragments.
Once the DNA is ready, it is phosphorylated, and an A-tail is added, before the adapter complex is ligated. This DNA-tail enables the DNA to bind to the flow cell which is covered with the reverse complement of the adapter (\autoref{fig:libraryprep}). 

\subsection{Next generation sequencing}
\label{intro-sec:ngs}

\begin{figure}[htp]
\centering
\includegraphics[width=.85\linewidth]{Figures/intro/SequencingBySynthesis.jpg}
\caption[Sequencing by synthesis (Illumina)]{The Illumina sequencing-by-synthesis approach. Cluster strands created by bridge amplification are primed and all four fluorescently labeled, 3'-OH blocked nucleotides are added to the flow cell with DNA polymerase. The cluster strands are extended by one nucleotide. Following the incorporation step, the unused nucleotides and DNA polymerase molecules are washed away, a scan buffer is added to the flow cell, and the optics system scans each lane of the flow cell by imaging units called tiles. Once imaging is completed, chemicals that effect cleavage of the fluorescent labels and the 3'-OH blocking groups are added to the flow cell, which prepares the cluster strands for another round of fluorescent nucleotide incorporation; Figure adapted from \protect\citeauthor*{Mardis2008}~\protect\cite{Mardis2008}}\label{fig:sequencingbysynthesis}
\end{figure}


Next-generation sequencing (NGS) is the coined term for basically any standard high-throughput sequencing performed, which includes exome, genome, transcriptome, \linebreak protein-\annote{DNA}{upper case} interactions (ChIP) and other epigenome studies. The term NGS is still widely used, even though it has been more than \change{10}{ten} years since the first NGS approach was commercially available. While at the beginning of next-generation sequencing, there were multiple approaches, the current lion's share (80\% of sequencing data) of protocols use the Illumina short read sequencing by synthesis approach (\autoref{fig:sequencingbysynthesis}) \cite{Mardis2008,Straiton2019}, which is based on the concept of alternating integration of fluorescently labelled nucleotides and imaging with a microscope (\autoref{fig:sequencingbysynthesis}), as well as multiplexing, where a DNA fragment is ligated to an index, which allows sequencing of multiple samples at once \cite{Church1984,Church1988}, as it is shown in \autoref{fig:libraryprep}. This method enables highly accurate determination of the sequence of a DNA fragment and, depending on the flow cell and sequencing machine, allows one to sequence a whole genome in just 24h.

\subsection[Long read sequencing]{Long read sequencing - the "third" generation sequencing}
\label{intro-sec:lrs}
By now, multiple methods \change{which}{that} broke free of the size limitations of NGS exist, \remove{which are }commonly referred to as long read sequencing. Most \remove{of the }current methods trade the very high accuracy of the second-generation NGS methods for the capability of sequencing huge continuous strands of DNA (current record 2.3 Million bp \cite{Payne2018}) with more typical library preparation ranging between 10-30 Kbp. 
These methods are expected to revolutionise our understanding of the highly repetitive elements \remove{that exist }in the genome, such as the centromeres of chromosomes. Methods such as the direct molecule sequencing approach by Oxford Nanopore \change{are even able to}{can even} distinguish post-transcriptional modifications on RNA \cite{Pratanwanich2021}.
However, for ctDNA, which is highly fragmented, these methods offer only limited advantages over \remove{the }short read sequencing.



% background of sequencing
\section[DNA analysis]{DNA analysis- what to do with the sequence}
\label{intro-sec:analysis}
The types of analysis that can be done with the output from the sequencing machine stretches far,  however, all methods need to first infer the location in the genome, the sequenced piece of DNA originated from. As the current methods randomly fragment the DNA (\autoref{intro-sec:libraryprep}), the genomic location information is completely lost. This process is referred to as mapping.

\subsection[Mapping]{Mapping - Ey man, where is my origin}
\label{intro-sec:mapping}
In this process, the fragments of DNA, which were sequenced, are assigned a genomic coordinate on the reference genome. This is only possible, due to the fact, that we have a resolved genome sequences (see~\autoref{intro-sec:sequencing}) for a high number of species. The location a sequenced piece of DNA fits to the reference genome might be unique, but it could also fit to multiple locations, due to highly repetitive regions or due to the existence of pseudo genes with almost 100\% identify. In addition to this, the reference genome might not accurately reflect the genome of the organism that has been sequenced. Each mapping position is therefore assigned a quality score, which reflects how likely it is the actual position of the sequence. As Illumina sequencers have the ability to sequence both ends of the DNA fragment, the position of the ends (read 1 and read 2) to each other can also be used to infer the quality, as they should be within a reasonable distance to each other (see \autoref{fig:libraryprep})

As this process is time consuming and the exact location of the fragment might not be as important, there exists a subset of tools called pseudo-mapper, which are based on $k$-mers, which are predefined DNA sequences of length $k$, which help to identify certain regions of interest. These tools are especially common for RNAseq, where the exact location of a read doesnt matter, only that the read is within a gene \cite{Bray2016,Patro2017}.

\subsection[Variant calling]{Variant calling - spot the distance}
\label{intro-sec:variantcalling}


% background of lung cancer
\section{Lungcancer}
\label{intro-sec:lungcancer}

% thesis overview
\section{Thesis overview and aims}
\label{intro-sec:overview}

While tumour heterogeneity is a well accepted concept by now, there is a need for computational methods assessing and visualising this heterogeneity. This work aims to contribute to this unmet need by developing three different custom made tools to infer and monitor tumour heterogeneity. I have completed the work in the following three parts:

\begin{enumerate}
	\item Development of two joint somatic variant calling workflows and the impact of these on downstream analysis (\autoref{ch:variantcalling}). 
	\item Analysis of five rapid autopsy probands with state of the art methods to investigate tumour heterogeneity and development of a mitochondrial based phylogeny reconstruction method (\autoref{ch:cascade}). 
	\item Development of a read-centric method to detect somatic mutational signatures from low coverage whole genome sequencing (\autoref{ch:mmf}).
\end{enumerate}
